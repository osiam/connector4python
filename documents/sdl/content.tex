\section{Introduction}

The audience of this document are members of the OSIAM-NG Team.
The purporse of this Document is to describe a process to enhance the quality of
OSIAM-NG from the perspective of security in iterative steps.

This document describes the SDL process which will fit into the short iterations of an agile development. It is splitted in three categories defined by frequency of completion.

\section{ Every-Sprint practices}

The first category consists of the SDL requirements that are essential to security \footnote{\url{http://www.blackhat.com/presentations/bh-dc-10/Sullivan_Bryan/BlackHat-DC-2010-Sullivan-SDL-Agile-wp.pdf} \url{http://www.microsoft.com/security/sdl/discover/sdlagile.aspx}}

\begin{itemize}
	\item Run static code analysis tools per build
	\item Define a list of approved tools and associated security checks
	\item Threat model all new features, determine risks from those threats, and establish appropriate countermeasures 
	\item Secure coding: cross-site scripting, interpreter injections(SQL, LDAP, JSON injections and so on), use only strong crypto, use filtering and escaping
	\item Secure design: attack surface reduction, principle of least privilege (a user should have only as many rights as it needs to do its job), secure defaults	
	\item Security Review includes an examination of threat models, tools outputs, and performance against the quality gates and bug bars. It results in one of three different outcomes: Passed , Passed with exceptions, failed  
\end{itemize}

\section{Bucket Requirements}
The second category of SDL requirements consists of tasks that must be performed on a regular basis over the lifetime of the project but that are not so critical as to be mandated for each sprint.
Instead of completing all bucket requirements each sprint, product teams must complete only one SDL requirement from each bucket of related tasks during each sprint.

\begin{itemize}
	\item Verification Tasks: Attack surface review, Negative Testing, regular Pen-Testing
	\item Design Review: Review crypto design, User Account Control
	\item Define/update Quality Gates/Bug Bar (A Bug Bar is a list of security vulnerabilities with then according severity: for example there can be a constraint that there will be no release if there are known vulnerabilities in the application with a ``critical'' or ``important'' rating)
	\item Identify functional aspects of the software that require closer review
	\item Deal with third-party software
\end{itemize}

\section{On requirement Changes}

The third category of SDL consists of task which must be done on every requirement changes which have consequences on the functional behaviour of the project itself. 

\begin{itemize}
	\item Security and privacy analysis includes assigning security experts
	\item Defining minimum security and privacy criteria for the application
\end{itemize}

The fourth category of SDL requirements consists of tasks that need to be met when you first start a new project or when you first start using SDL with an existing project. These are generally once-per-project tasks that won’t need to be repeated after they’re complete.

\section{One-Time Requirements}
\begin{itemize}
	\item Deploying a security vulnerability item tracking system which allowes for creation, tracking and reporting of software vulnerabilities
\end{itemize}

\section{Third-Party Software}
Third-Party software can introduce its own security vulnerabilities and effective countermeasures can only be constructed if one is aware of them. Bugtrackers which exist for most software projects can be used to keep an eye on such vulnerabilities but have to be checked regularly. This can easily grow out of hand if a project accumulates a very large number of thid-party dependencies and checking each one for security issues can be a an insurmountable task. In such a case it is necessary to select key dependencies and to focus on those.

\section{Testing}

The testing process cannot be described right now because we haven't done any security related tests yet. 

\section{Handling the Bug Bar}

The handling of the Bug Bar cannot be described because we don't have any bug tracker, yet.
