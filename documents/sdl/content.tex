\section{Introduction}

The audience of this document are members of the OSIAM-NG Team.
The purporse of this Document is to describe a process to enhance the quality of
OSIAM-NG from the perspective of security in iterative steps.

\section{Process}
This chapter describes the SDL process which will fit into the short iterations of an agile development. It is splitted in three categories defined by frequency of completion.

The first category consists of the SDL requirements that are essential to security \footnote{\url{http://www.blackhat.com/presentations/bh-dc-10/Sullivan_Bryan/BlackHat-DC-2010-Sullivan-SDL-Agile-wp.pdf} \url{http://www.microsoft.com/security/sdl/discover/sdlagile.aspx}}

1. Every-Sprint practices
\begin{itemize}
	\item Run static code analysis tools per build
	\item Define a list of approved tools and associated security checks
	\item Threat model all new features, determine risks from those threats, and establish appropriate countermeasures 
	\item Secure coding: cross-site scripting, interpreter injections(SQL, LDAP, JSON injections and so on), use only strong crypto, use filtering and escaping
	\item Secure design: attack surface reduction, principle of least privilege (a user should have only as many rights as it needs to do its job), secure defaults	
	\item Security Review includes an examination of threat models, tools outputs, and performance against the quality gates and bug bars. It results in one of three different outcomes: Passed , Passed with exceptions, failed  
\end{itemize}

The second category of SDL requirements consists of tasks that must be performed on a regular basis over the lifetime of the project but that are not so critical as to be mandated for each sprint.
Instead of completing all bucket requirements each sprint, product teams must complete only one SDL requirement from each bucket of related tasks during each sprint.

2. Bucket Requirements
\begin{itemize}
	\item Verification Tasks: Attack surface review, Negative Testing, regular Pen-Testing
	\item Design Review: Review crypto design, User Account Control
	\item Define/update Quality Gates/Bug Bar (A Bug Bar is a list of security vulnerabilities with then according severity: for example there can be a constraint that there will be no release if there are known vulnerabilities in the application with a ``critical'' or ``important'' rating)
	\item Identify functional aspects of the software that require closer review
	\item Deal with third-party software
\end{itemize}

The third category of SDL consists of task which must be done on every requirement changes which have consequences on the functional behaviour of the project itself. 

3. On requirement Changes

\begin{itemize}
	\item Security and privacy analysis includes assigning security experts
	\item Defining minimum security and privacy criteria for the application
\end{itemize}

The fourth category of SDL requirements consists of tasks that need to be met when you first start a new project or when you first start using SDL with an existing project. These are generally once-per-project tasks that won’t need to be repeated after they’re complete.

4. One-Time Requirements
\begin{itemize}
	\item Deploying a security vulnerability item tracking system which allowes for creation, tracking and reporting of software vulnerabilities
\end{itemize}


\section{List of possible targets}
In OSIAM-NG there are several possible targets for an attacker:
\begin{itemize}
\item REST-APIs -- an attacker can use our REST-API to gain informationen about or overload the system or application.
\item Running Container -- an attacker can use informations about the running container (e.q. Tomcat) to start a specific attack.
\item Database -- an attacker could obtain access to the used database and gain sensible data.
\item Running System -- an attacker could try to attack the system directly to gain access.
\end{itemize}

\subsection{API Calls}
\subsubsection{SCIM}
Every call of SCIM is secured with OAuth2, so, you need a valid access-token to gain access. We define Java-Script-, SQL- and JSON-Injections as 'Injection'.

\paragraph{GET authorization-server/[User|Group]/\{ID\}}

Response is an Error- or Resource in JSON.

\begin{tabular}{|l|l|}
    \hline
    DDoS & Block IP after n successless tries\\
    \hline
    ID Guessing & generalize Error-Response and block IP after n successless tries\\

    \hline
 \end{tabular}

D
g
Resource as JSON\paragraph{POST|PUT|PATCH authorization-server/[User|Group]/\{ID\} Resource as JSON} 

Response is an Error- or Resource in JSON.

 
\begin{tabular}{|l|l|}
    \hline
    Injection & Whitelist Input\\
    \hline
    JSON-Parser overload & Parser Testen \\
    \hline
    insecure Database Connection & encrypt sensible Data \\
    \hline
 \end{tabular}

\paragraph{DELETE authorization-server/[User|Group]/\{ID\}}

Response is an HTTP-Statuscode.

\begin{tabular}{|l|l|}
   \hline
    ID Guessing & block IP after n successless tries\\
    \hline
 \end{tabular}

\subsubsection{OAuth2}

\paragraph{Auth-Code flow}

Response is an Auth-Code or an Error.

\begin{tabular}{|l|l|}
    \hline
    unknown redirect URI & do not redirect User back to unknown redirect URI\\
    \hline
    unsecure credentials & hash and salt sensible credentials \\
    \hline
    insufficient user rights & control the right of an user before granting access \\
    \hline
    error response is too verbose & generalize error response \\
    \hline
    unsecure connection between client and authz & use TLS \\
    \hline
    no logout & implement logout \\
    \hline
    unable to revoke given grants & implement functionality to revoke given grants \\
    \hline
 \end{tabular}

\paragraph{Access-Token flow}

Response is an Accesstoken or an Error.

\begin{tabular}{|l|l|}
    \hline
    wrong redirect URI & sent error response\\
    \hline
    intercept client secret or access-token & use TLS \\
    \hline
    access\_token does not expire & do not automatically refresh access\_token \\
    \hline
 \end{tabular}

\section{Type of Attacks}
SQL Injection, Java-Script injection via User-Input.
Guessing of IDs via error messages, DDoS, 
attacks to guess a password by measuring response time
\subsection{Destroy the System}
SQL Injection, DDoS, ...

\subsection{Leak Informations}
SQL Injection, Java-Script injection, Guessing of IDs via Error-Messages, 
attacks to guess the password by time measuring, attacks on the Database to retrieve Informations.

\subsection{Corrupt the User-Agent}
Java-Script injection, XSS,...

\subsection{Technical}
%https://www.owasp.org/index.php/SQL_Injection_Prevention_Cheat_Sheet
%https://www.owasp.org/index.php/XSS_%28Cross_Site_Scripting%29_Prevention_Cheat_Sheet
%https://www.owasp.org/index.php/XSS_Filter_Evasion_Cheat_Sheet
%https://www.owasp.org/index.php/Session_Management_Cheat_Sheet
%https://www.owasp.org/index.php/Logging_Cheat_Sheet
%https://www.owasp.org/index.php/Cheat_Sheets

%Security related newsgroups:
%http://www.securityfocus.com/
%http://www.securityweek.com/
%OWASP Guide zu allen Bereichen --> https://www.owasp.org/index.php/Guide_Table_of_Contents
%OWASP Testing Guide --> https://www.owasp.org/index.php/OWASP_Testing_Guide_v3_Table_of_Contents
%OWASP App Security Tutorials --> https://www.owasp.org/index.php/OWASP_Appsec_Tutorial_Series
%OWASP Secure Coding Practices --> https://www.owasp.org/index.php/OWASP_Secure_Coding_Practices_-_Quick_Reference_Guide

%Tools:
%OpenSAMM: Model, Templates, Sheets --> http://www.opensamm.org/
%OWASP LAPSE (Security scanner for Java EE Applications)
%OWASP Zed Attack Proxy (ZAP) is an easy to use integrated penetration testing tool for finding vulnerabilities in web applications
%SDL Threat Modeling Tool (Windows only???)

%Security Books:
%Writing Secure Code, 2nd Ed by David LeBlanc and Michael Howard					 
%19 Deadly Sins of Software Security by Michael Howard, David LeBlanc, and John Viega		 	 		
%Building Secure Software by John Viega and Gary McGraw		 		 	
%Gray Hat Hacking by Shon Harris, et al.	 		 		
%How to Break Software Security by James Whittaker and Herbert Thompson
