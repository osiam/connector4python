\section{Introduction}

The audience of this document are members of the OSIAM-NG Team.
The purporse of this Document is to describe a process to enhance the quality of
OSIAM-NG from the perspective of security in iterative steps.

\section{Threat Model}
Will be a Picture.

\section{List of possible Target}
In OSIAM-NG are several possible targets for an attacker:
\begin{itemize}
\item REST-APIs -- an attacker can use our REST-API to gain informationen about or overload the system or application.
\item Running Container -- an attacker can use informations about the running container (e.q. Tomcat) to start a specific attack.
\item Data Base -- an attacker could obtain access to the used database and gain sensible data.
\item Running System -- an attacker could try to attack the system directly to gain access.
\end{itemize}

\subsection{API Calls}
\subsubsection{SCIM}
Every call of SCIM is secured with OAuth2, so, you need a valid access-token to gain access. We define Java-Script-, SQL- and JSON-Injection as 'Injection'.

\paragraph{GET authorization-server/[User|Group]/\{ID\}}

Response is a Error- or Resource in JSON.

\begin{tabular}{|l|l|}
    \hline
    DDoS & Block IP after n successless tries\\
    \hline
    ID Guessing & generalize Error-Response and block IP after n successless tries\\

    \hline
 \end{tabular}

\paragraph{POST|PUT|PATCH authorization-server/[User|Group]/\{ID\} Resource as JSON} 

Response is a Error- or Resource in JSON. 
\begin{tabular}{|l|l|}
    \hline
    Injection & Whitelist Input\\
    \hline
    JSON-Parser overload & Parser Testen \\
    \hline
    insecure Database Connection & encrypt sensible Data \\
    \hline
 \end{tabular}

\paragraph{DELETE authorization-server/[User|Group]/\{ID\}}

Response is a HTTP-Statuscode.

\begin{tabular}{|l|l|}
   \hline
    ID Guessing & block IP after n successless tries\\
    \hline
 \end{tabular}

\subsubsection{OAuth2}

\paragraph{Auth-Code flow}

Response is a Auth-Code or a Error.

\begin{tabular}{|l|l|}
    \hline
    unknown redirect URI & do not redirect User back to unknown redirect URI\\
    \hline
    unsecure credentials & hash and salt sensible credentials \\
    \hline
    insufficient user rights & control the right of an user before granting access \\
    \hline
    error response is too verbose & generalize error response \\
    \hline
    unsecure connection between client and authz & use TLS \\
    \hline
    no logout & implement logout \\
    \hline
    unable to revoke given grants & implement functionality to revoke given grants \\
    \hline
 \end{tabular}

\paragraph{Access-Token flow}

Response is a Accesstoken or a Error.

\begin{tabular}{|l|l|}
    \hline
    wrong redirect URI & sent error response\\
    \hline
    intercept client secret or access-token & use TLS \\
    \hline
    access_token does not expire & do not automatically refresh access_token \\
    \hline
 \end{tabular}



\subsubsection{Root}
Comming soon.

\subsubsection{User}

\subsubsection{Group}

\section{Type of Attacks}
SQL Injection, Java-Script injection via User-Input.
Guessing of IDs via error messages, DDoS, 
attacks to guess a password by measuring response time
\subsection{Destroy the System}
SQL Injection, DDoS, ...

\subsection{Leak Informations}
SQL Injection, Java-Script injection, Guessing of IDs via Error-Messages, 
attacks to guess the password by time measuring, attack the DataBase to retrieve Informations

\subsection{Corrupt the User-Agent}
Java-Script injection, XSS,...

\section{Possible Counter Actions}

\subsection{Process}
Security-Audits, regular Pen-Testing

\subsection{Technical}
%https://www.owasp.org/index.php/SQL_Injection_Prevention_Cheat_Sheet
%https://www.owasp.org/index.php/XSS_%28Cross_Site_Scripting%29_Prevention_Cheat_Sheet
%https://www.owasp.org/index.php/XSS_Filter_Evasion_Cheat_Sheet
%https://www.owasp.org/index.php/Session_Management_Cheat_Sheet
%https://www.owasp.org/index.php/Logging_Cheat_Sheet
%https://www.owasp.org/index.php/Cheat_Sheets

\section{How to Find Security Leaks}


\subsection{Penetration Testing}

%https://www.owasp.org/index.php/Web_Application_Security_Testing_Cheat_Sheet

\subsection{Tools for automatic Testing}
(e.q. UI Tests)

%https://www.owasp.org/index.php/Appendix_A:_Testing_Tools

\subsection{How to Deal with Dependencies in our Software}
\subsection{How to Deal with the Infrastructure}
\section{Points to Deal within the Future}
